\chapter{A Social Contract for Cyberspace}
\section{Introduction}
This chapter addresses a critical
weakness in the whole concept on which the dissertation is based. It
addresses a crucial issue on which they all depend: ``How can stakeholders
agree with each other about what is allowed in each other's policies?''

In our world, the ethics of humans can be different from one to another,
without the necessity for conflict. There are, probably, some ethical
principles that are ``universal'', i.e. we all share them. Respect for life,
for property, But there are other values (ethics, or agreed principles of behaviour in general) 
that are not universal. The fact that human values vary from person to person does not obviate
the need to incorporate human and ethical values (and principles
of operation in general) into software. According to this dissertation's point of view, the
key to achieving this goal, without forcing all users to adopt ethical
principles that they don't agree with, is to seek agreement only on a
core set of ethical principles: the social contract.

It might appear that the proposed social contract in this chapter is not a
technical concept but rather, merely, an appeal to the good behaviour of
citizens of cyberspace. However, the authors in \cite{sheniar2021social}
expect that technical means for enforcing policies will become more
widespread, leading to an increasingly rigorous, and technical, role
for the social contract.
