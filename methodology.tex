\chapter{Methodology}
\section{Introduction}
Cyberspace is the digital creativity environment in which we live and which is controlled by cybersecurity.  Perhaps the basic vision of cybersecurity is to reach a \emph{virtuous} digital world.  In order to reach such a strategic goal, the mission of cybersecurity must be how to insert human ethics and values into software values.  This matter in itself is not easy at all because it requires an ethical philosophy related to cybersecurity.  This philosophy proposes that cyberspace be an \emph{ideal} alternative to the real world as much as possible.  What we mean by \emph{idealism} here is that the presence of each member in this space is effective and not harmful to others.  To achieve this requires answering two important questions.
\begin{enumerate}
\item What is the current presence of members in cyberspace? And,
\item how can this presence be digitally enhanced to be ideal?
\end{enumerate}
      Through this chapter, we tried to provide a convincing answer to both questions by developing a  logical framework for cybersecurity. %that proposes stakeholder policies and a social contract of cyberspace.

\section{Logical analysis}

\section{Experiments using examples}

\section{Gathering ideas and reviewing ideas from literature}

\section{Stakeholder Security Analysis}
In this section, we designed stakeholders' policies to incorporate human
values and ethics as much as possible into software values

\iffalse
\section{A social contract for cyberspace}
We proposed a social contract for cyberspace that can manage Stakeholders'
policies and ensure their respect by their owners.
\fi

