\chapter{Methodology}
\section{Introduction}
Cyberspace is the digital creativity environment in which we live and
which is controlled by cybersecurity.  Perhaps the basic vision of
cybersecurity is to reach a \emph{virtuous} digital world.  In order to
reach such a strategic goal, the mission of cybersecurity must be how
to insert human ethics and values into software values.  This matter
in itself is not easy at all because it requires an ethical philosophy
related to cybersecurity.  This philosophy proposes that cyberspace
be an \emph{ideal} alternative to the real world as much as possible.
What we mean by \emph{idealism} here is that the presence of each member
in this space is effective and not harmful to others.  To achieve this
requires answering two important questions.

\begin{enumerate}
\item What is the current presence of members in cyberspace? And,
\item how can this presence be digitally enhanced to be ideal?
\end{enumerate}

Through this chapter, we tried to provide a convincing answer to both
questions by developing a  logical framework for cybersecurity. 
%that proposes stakeholder policies and a social contract of cyberspace.



\section{Logical analysis}

[One of the methods used in this work is to apply logic to cybersecurity objectives. 

In particular, logic can be used to show {\em how} cybersecurity objectives can
be met. Logic is the appropriate tool, for example, if the objective sought can
be shown to follow, logically, from certain other rules.

In some cases it may also be possible, and useful, to show that certain objectives can {\em only}
be met in certain ways.]

\section{Experiments using examples}

[It is not always possible to prove hypotheses, even when there are strong reasons for believing
them to be true. In science, and even in mathematics, some valid hypotheses can only be demonstrated
by examples. A very good example of this sort of hypothesis is Church's thesis, which is that
all concepts of computability are equivalent to Turing computability. A very large number of papers
were written, in the past, which showed that this or that type of computability was equivalent to
Turing computability. Eventually, journals put a stop to such publications. The concept that there is
really just one type of computable function had been established. But it was not a provable concept.
It was necessary to prove it by examples.

There are other advantages of examples. They are also an ideal way to demonstrate ideas in a 
natural and attractive fashion.

In this dissertation, many examples will be used to illustrate ideas which are in an obvious
sense quite similar in multiple superficially different cases. The striking similarity, in
different contexts, demonstrates that the concept is not an accident, but represents an important
principle.]

\section{Gathering ideas and reviewing ideas from literature}

[An important methodology in this dissertation, as in all scientific work, is to gather the thoughts
of other researchers, who are struggling with the same problems as tackled here, and to identify
key ideas and principles that have previously been suggested, and to recognise the most important
of these and make use of them in this work.]

\section{Stakeholder Security Analysis}

[Stakeholder security analysis is a more specific form of logical analysis. Instead of dealing with
objectives in general, it makes use of the obvious but useful fact that all systems have
stakeholders, and the objectives of the system as a whole can be sub-divided into those
which can be ascribed to each stakeholder. Once their objectives have been identified, they
can be further analysed, refined, and reviewed. In particular, it is important to work out if
their objectives are compatible.]

In this section, we designed stakeholders' policies to incorporate human
values and ethics as much as possible into software values

\iffalse
\section{A social contract for cyberspace}
We proposed a social contract for cyberspace that can manage Stakeholders'
policies and ensure their respect by their owners.
\fi

