\chapter{Methodology}
\section{Introduction}
Cyberspace is the digital creativity environment in which we live and which is controlled by cybersecurity.  Perhaps the basic vision of cybersecurity is to reach a \emph{virtuous} digital world.  In order to reach such a strategic goal, the mission of cybersecurity must be how to insert human ethics and values into software values.  This matter in itself is not easy at all because it requires an ethical philosophy related to cybersecurity.  This philosophy proposes that cyberspace be an \emph{ideal} alternative to the real world as much as possible.  What we mean by \emph{idealism} here is that the presence of each member in this space is effective and not harmful to others.  To achieve this requires answering two important questions.
\begin{enumerate}
\item What is the current presence of members in cyberspace? And,
\item how can this presence be digitally enhanced to be ideal?
\end{enumerate}
      Through this chapter, we tried to provide a convincing answer to both questions by developing a  logical framework for cybersecurity. %that proposes stakeholder policies and a social contract of cyberspace.

\section{Logical analysis}

\section{Experiments using examples}
\iflonger
\subsection{security audit}
An approach used periodically is to assign the task of attack or harm   a service in cyber space  to an individual or a team and then address the discovered weaknesses. We call this a {\em security audit}. The approach of searching for weaknesses and fixing them is so widely used that it might reasonably be regarded as a design philosophy.
This approach has been used in this dissertation, and the results of
both the attacks and the resulting defences are reported.
However, this approach is somewhat pessimistic in that it {\em assumes}
that a more methodical security design philosophy which can guarantee
secure design is not available.
\fi

\subsection{Web Service Security Design}\label{webservicessecurity}
Because web services (including services provided via apps on mobile phones)
are a recent development and continue to evolve in both details and fundamentals,
principles of secure design of these services is also a new and evolving area of
research and development \iflonger\cite{AddieColman2010,Addie_Moffatt_Dekeyser_Colman2011}.
\else\cite{Addie_Moffatt_Dekeyser_Colman2011}.\fi

This section reviews three different approaches for securing web sites/services.
Each of these approaches is usually expressed as a completely independent
philosophy for achieving good security. We shall see that these approaches 
are actually complementary, and to achieve rigorous security all three approaches
are needed. Note that although we describe a design philosophy which is able, formally, to prove,
i.e.  guarantee, security, because no logical system can claim certainty in an absolute
sense (in mathematical logic, this fact is expressed in G\"odel's incompleteness theorem), the strategy 
of attacking the system remains useful, even after it has been
methodically proved to be correct.

The present paper does not apportion equal emphasis on all approaches
because the original contribution of this paper is 
in the third of them, together with the way the second approach joins
with the third to form a more comprehensive whole. The second approach
is the one summarised in Subsection \ref{audit} and applied
in Section \ref{expts}.
The third approach is summarised in Subsection \ref{stsecanal} and applied
to the Netml password reset system in Sections \ref{stakeholders} and \ref{proof}.


\subsubsection{Good Security Design Practice}
Good design takes security, ease of access, and usability into account,
striking a balance between protecting the system and ease of use. Good
practice has evolved a number of practical approaches like  minimizing
attack surface area \cite{bhardwaj2018reducing},  establish secure
defaults \cite{lai2018impact}, using the principle of defence in depth
\cite{toch2018privacy}, not trusting services \cite{ghirardello2018cyber},
keeping security simple \cite{thomsen2018network}, and fixing security
issues correctly \iflonger\cite {ali2018security, tabassum2018evaluating}.
\else \cite {tabassum2018evaluating}.\fi
These approaches  are used for  maintaining and improving security which
are so natural and important that they should be adopted as a first
layer of protection as a matter of standard practice, even when more
sophisticated approaches are also in use \cite {ross2018systems}.


\subsubsection{Security Auditing}\label{audit}

Strategies for breaking into web systems or services are under continuous
development by government and non-government organisations and individuals,
both those with friendly intentions and those who wish to exploit security
weaknesses for their own advantage. When
a new exploit is discovered, if it is discovered first by those with 
friendly intentions, defences against the exploit are usually developed
quickly and published. Exploits discovered by attackers with ill intent
can, of course, be deployed before web managers have the opportunity
to defend against them. Also, in the period of time immediately after
the defence against a new exploit has been published, there is still
an opportunity to attack web sites which have not deployed the newly developed
defences. This time can be somewhat extended due to the limitated expertise
of web-site owners and because the sequence of steps required to address a weakness
in a high-level framework can be quite lengthy.

A widely used strategy for improving web site or web service security is
to attempt to attack the site by using the strategies which are currently
known to be effective one-by-one, or simultaneously, to discover if the
site is vulnerable to any of these strategies. Since all of the strategies
tried are known, the defences against all of them are also almost certainly
known, and hence can be adopted by the web site.

Experiments of this type, and the resulting web service design improvements,
are described in Section \ref{expts}.

\subsubsection{Stakeholder Security Analysis}\label{stsecanal}

A more fundamental strategy which is not well-developed at present is
to seek to develop provably secure protocols and software for all aspects
of a web service \cite{whitman2011principles,mailloux2018examination,bishop2005introduction}.

The first step in this approach, which is developed further in 
Section \ref{stakeholders}, is to consider the point of view of all legitimate stakeholders
in relation to the service, and to enumerate a complete set of rules
required by each of these stakeholders, sufficient to ensure that they
will agree to actively participate in the service.
\subsubsection{Stakeholder Roles}\label{strol}
The roles are defined first, then their stakeholders, and access levels and features are set in those roles. Security rules are defined based on stakeholders roles, and not stakeholders themselves, where we can state the rule in terms of anyone who is playing a certain role. Stakeholder experiences should be shared to identify vulnerabilities that may occur while dealing with the service or part of the system under study. Most organizations consider the reputable more than the performance of their systems. Therefore, they adopt robust security solutions. However, access levels for sources as well as sensitive data often require additional restrictions above standard security requirements. Hence,  additional restrictions are identified and chosen with extreme care, ensuring their independence, and providing sources and data to the real interest without any other party who can access the current security restrictions.
\subsubsection{Stakeholder Business Rules}\label{stbsnrul}
Business rules refer to any rule relating to the way in which transactions are restricted. Business rules must be observed when designing and improving system security. In other words, even if the proposed security rules are enforced, they must be considered to be subject to the satisfactory standard by their professional organization. It is necessary to focus on Business rules  which are explicit rules on safety and try to study them and analyze them before drafting the security rules to avoid the contradiction.
The expectations of stakeholders which constitute a security case are considered mandatory and must be included in a security rule

\subsubsection{Stakeholder SWOT}\label{stswot}
The SWOT analysis tools are used to evaluate the security aspects for the stakeholders in the system Fig(). The first step is to determine the security objectives required by each stakeholders role and business rules within the specific service,Password Reset Service. And then identify the internal factors (strengths, weaknesses) and external factors (opportunities, threats) that are favourable and unfavourable to achieve those objectives.

Strengths are an opportunity to support security objectives, while weaknesses are considered to be the vulnerabilities that may impede the achievement of these objectives. The statement of the risks of weaknesses and the great efforts to fix them and clarify the strengths in the implementation of the security rules that prevent such circumstances are sufficient justification for obtaining the approval of the owners of the decision. On the other hand, the lack of funds and inconsistencies with business rules in achieving security objectives are weaknesses. In terms of opportunities, they are any initiatives available to support security objectives such as sources, expenditures, technologies, and security policies.Threats are related to the importance of the system and the sensitive information it contains. They often exploited weaknesses.

%These objectives support the protection of password reset service.
\subsubsection{Stakeholder Security Rules}\label{stsecrul}
\subsubsection{Assumption}\label{assmp} 
After completing security rules, assumptions are developed that simplify the logic necessary to complete the design process. The degree of simplification is determined depending on the sensitivity of the security issue. Security rules are therefore classified as optional or mandatory. Unlike the optional category, which is assumed or obvious assumptions, mandatory class assumptions are security requirements that need to be adopted in the design or development of system security.
\subsection{Stakeholder Security Design}\label{stsecdsgn} 
The main goal in the design phase is to know which of the mandatory security rules must be guaranteed and how to do it. When a particular  rule is not guaranteed directly,  this rule should be checked  that it follows from our assumptions and the rules which are enforced. However , there are three categories of rules: (i)assumptions; (ii) enforced rules; (iii)rules that must be proven, and through careful examining of the latter, the process of checking security becomes more reliable and systematic.

In addition, unrealistic assumptions may cause security vulnerability. Also, very strict assumptions can undermine security. However, the combination of the two methods can lead to simple and effective security design and achieve a higher security level in certain situations of critical importance.
\subsubsection{Inforcable Stakeholder Security Rules}\label{Instsecrul}
\subsubsection{Validation of Stakeholder Security Rules}\label{vlstsecrul}

\section{Gathering ideas and reviewing ideas from literature}

\section{Stakeholder Security Analysis}


The main result of this dissertation is to demonstrate  a secure methodical design philosophy.
We call this {\em stakeholder security analysis}.

Stakeholder security analysis proceeds as follows: 
\begin{enumerate}[1.]
\item Identify the key stakeholders.
\item For each stakeholder, identify a set of rules {\em required}
by these stakeholders. Note: the collected rules required
by all stakeholders must be consistent.
\item Implement procedures which ensure that all rules are enforced.
\end{enumerate}

Both a security audit and a stakeholder security analysis
are applied in this dissertation to a specific subsystem of a web
service system being developed and managed by the authors.
Note: it is not suggested that stakeholder security analysis obviates
the need for a security audit.

\iffalse
\section{A social contract for cyberspace}
We proposed a social contract for cyberspace that can manage Stakeholders'
policies and ensure their respect by their owners.
\fi

