\chapter{INTRODUCTION}
% \minitoc

\if 
This research aims to propose sophisticated design methodologies to enhance cybersecurity architecture through:
\begin{itemize}
\item Improving stakeholders cybersecurity requirements by using stakeholder's cybersecurity analysis.
\item Secured accessing
the cyber-physical systems (the isolated systems) by using the cybersecurity of targeted entity methodology.
\item Measuring the accuracy of believing for the true level of published news by using the cybersecurity reference classes methodology. 
\end{itemize} \fi
\if
This research aims to propose,  demonstrate, and develope a Cybersecurity Architecture of Related Entities through:
\begin{itemize}
\item Stakeholder's cybersecurity  %by using stakeholder's cybersecurity analysis.  
\item  The cyber-physical systems %by using the cybersecurity of targeted entity.
\item  News Checking %by using the cybersecurity reference classes. 
\end{itemize}
\fi


\section{Research Background}
\subsection{Cybersecurity Definition}
\subsection{Cybersecurity Policies}
\subsection{Mantodory Objectives}
\subsection{Social Contract}

\section{Significance of the Research}
\section{Research Problem}
According to the philosophical and mathematical viewpoint of cybersecurity in this research:

Cyberspace can be considered as a network of heterogeneous entities
(individual users, organizations, devices, service providers, and
services etc.). These entities can be classified into two main categories,
the services, and their stakeholders (cyberspace diagram).  Stakeholders
can set their own policies regarding the services they engage with. To
assist them in this setting, they can use the policies declared by
those services.

Stakeholders polices are used to achieve their key (essential)
objectives. Different stakeholders have different objectives and also,
their objectives are subject to change.  For the stakeholders to be
able to work together effectively, their policies must be consistent.
Furthermore, as stakeholders objectives and policies change over time,
this consistency must be maintained.

Stakeholder objectives and policies are designed to maintain safety, privacy, security,
and any other key objectives. Any inconsistency between policies will lead
to a breakdown or compromise of the overall system.

In order for stakeholders to maintain their essential objectives, it is
generally necessary to develop a collection of enforcable conditions and to configure these conditions so that they achieve the objectives. This process is sometimes referred to as policy refinement. [Find citations in James Northway's
proposal.]

A social contract is needed to enable and manage these conditions (consistency diagram).  While ensuring the achievement of stakeholders’ objectives, the consistency of all the policies is essential, but this will be difficult unless stakeholders have a sound basis of shared {\em values}. The social contract provides this basis.
\section{Research Question(s)}
\if
What is the logical cybersecurity framework within which all the key concepts are consistent?\\
What is the cybersecurity framework within which all the key concepts are consistent?\\
What is the cybersecurity logical consistency framework within which all the key concepts are fit?\\
\fi
%What is the logical cybersecurity framework within which all the key concepts are consistent?\\

What is the logical process to achieve cybersecurity within which all the key concepts are consistent?
\begin{enumerate}
\item Why is a social contract needed and how can one be proposed for cyberspace?
\item How can we develop and configure a collection of enforceable conditions to maintain safety, privacy, security, and any other key objectives?
\item How can we design, and implement software which enforces the social contract.% and ensure that everyone installs thissoftware and uses only this software in the course of their interaction
\end{enumerate}



\section{Research Aims and Objectives}
Cyber security policies should be produced within a teamwork approach.  This partnership needs a logical framework that controls these policies through a set of conditions that guarantee the achievement of the mandatory objectives.  The social contract comprehensively controls how these policies operate together, in other words, enables and manages these conditions.  This contract should be approved by the stakeholders. Finally, we develop, design, and implement software that enforces this contract, and ensure that this software is dealt with exclusively by all parties.

\section{Research Limitations}
