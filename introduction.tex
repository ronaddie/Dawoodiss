\chapter{Introduction}
% \minitoc

\if 
This research aims to propose sophisticated design methodologies to enhance cybersecurity architecture through:
\begin{itemize}
\item Improving stakeholders cybersecurity requirements by using stakeholder's cybersecurity analysis.
\item Secured accessing
the cyber-physical systems (the isolated systems) by using the cybersecurity of targeted entity methodology.
\item Measuring the accuracy of believing for the true level of published news by using the cybersecurity reference classes methodology. 
\end{itemize} \fi
\if
This research aims to propose,  demonstrate, and develope a Cybersecurity Architecture of Related Entities through:
\begin{itemize}
\item Stakeholder's cybersecurity  %by using stakeholder's cybersecurity analysis.  
\item  The cyber-physical systems %by using the cybersecurity of targeted entity.
\item  News Checking %by using the cybersecurity reference classes. 
\end{itemize}
\fi


\section{Research Background}
\subsection{Cybersecurity Definition}
In recent years, the term cyber security has been widely used \cite{craigen2014defining,schatz2017towards,giles2013divided}, however, because of its recent origin, there are differing interpretations of its meaning. Also, current interpretations of this term fluctuate quite dramatically for two reasons \cite{schatz2017towards}. The first is the diversity of sources of definition between governmental, professional and academic domains. The second  is the rapid development, especially in the field of technology, which sometimes leads to a change in the inpretation of key terms. This diversity in assumed definitions leads to a difference of views on what cyber security is and what are the related cybersecurity and ethical issues. It is also likely to cause significant problems in the context of organizational strategies, stakeholder policy consistency, business objectives, and international agreements. 

In this research, we review, in Chapter 2,  the existing literature to determine the definitions provided for the term cybersecurity by in current literature. We now  propose a new definition, with considerable advantages in clarity and simplicity, 
which is more appropriate for the research reported in this dissertation. Alternative interpretations, or definitions, of cybersecurity will be considered in more depth in Chapter 2.  The definition we now give is somewhat broader than is often adopted
but it has advantages of applicability and logical consistency which
we hope the reader will appreciate when it has been tested in a 
variety of contexts.

\subsubsection*{New Definition}
\begin{quote} \em Cybersecurity is the logical process, including design and implementation, of ensuring that one's mandatory objectives are met, where mandatory means the objectives which one wishes to be certain of achieving.\end{quote}

This definition of cybersecurity includes consideration of privacy, safety, and financial risks, ...
anything that is mandatory. This view is adopted to allow the different mandatory goals
to be designed in a fully coordinated manner, which is logically necessary.

\subsection{Cybersecurity Policies}
%The discrepancy in the current cybersecurity definitions and the difference in objectives lead to significant problems with policies consistency. Any inconsistency will lead to not being able to work together effectively.
All devices, servers, agents, and even individual applications, running in a computer-based device have an explicit or implicit {\em policy} which governs what this particular agent or process {\em permits} to occur, on its account. In all major operating systems policies, for all installed applications, policies are already or soon will be an important component of system security.
The current policies may support the practice 
of unethical issues through cyberspace.
 These policies seek to achieve certain mandatory goals without the other goals.
%Like this approach in itself led to the creation of a safe environment for the practice of unethical issues.
For example,  that surface web focused on transparency
in tracking information sources to the extent that the privacy of
individuals and institutions was violated by government agencies or
rights holders. On the other hand, on the deep Internet, this privacy
is preserved in Tor, to the extent that this
privacy is a starting point for the creation of a virtual crime world
\emph{Dark Side} \cite{gupta2021dark,kavallieros2021understanding}. It seems that the lack of consistency between the
primary objectives is a clear weakness in cybersecurity policies.

Stakeholders need to ''negotiate'' their policies with each other in
order to make their own and fashion others' policies. This process of
negotiation of policies needs some guidelines or rules. We have proposed,
in particular, that all parties accept a "social contract" which is,
primarily, that all stakeholders honour their own policies. In the
process of this negotiation, stakeholders should constantly check the
coherence of the policies of their peers, ie of the systems of agents
with whom they work.

It is clear that all stakeholders want to achieve all their objectives,
not just security objectives, not just safety objectives, etc. So, if
there is any logical question about objectives that we need to address,
they must be dealt with as a whole, not separately. Eg if some other
agent's safety policy is inconsistent with our security policy, that will
be important for us to know, and we must take it into account. This
shows that any method for designing policies, and in particular,
any method which seeks to ensure that the policies of a collection of
agents are consistent, must make use of each agent's entire policy,
not just a sub-policy in one sense (eg safety) or another. It seems
that safety, privacy, and in fact all objectives must be taken into
account simultaneously.
\subsection{Human ethics and values}
Fundamentally, human values and ethics should, or must, take precedence in our design of cybersecurity. The reason is that the whole system -- cyberspace -- is a human system. If it doesn't serve the real needs of humans, it must be changed so that it does. 

However, One of the most important challenges facing cybersecurity is the insertion of human values into software values.   
The more deficient this system is in transforming these values and ethics, the more the resulting software solutions are dependent on the human factor (stakeholders). Such solutions cannot always be trusted. For example, it cannot be guaranteed that governments will not exploit the transparency of tracking in search engines to pursue opponents of their political approach, nor can it be guaranteed that people will not exploit the absence of such transparency to commit crimes or trade illegally through Deeb web. 

In this dissertation, we have tried to develop a logical framework for cyber security that consists of two main stages. The first is through which human values and ethics are transformed as much as possible into software values through the design of stakeholder policies. The second is to propose a social contract for cyberspace that can manage these policies and ensure their respect by their owners

By recognizing this and taking it into account, we will make cyberspace much better, especially,  to support the rapid technological development and the increasing online services and to curb cybercrime practices, immoral issues and illegal trade.

\subsection{Logic of Cybersecurity}
 
Cybersecurity is of little value unless it is correct. In many fields,
including cybersecurity, learning the right rules, implementing
these rules, and making sure that they are maintained, is considered
a satisfactory working procedure. However, as technology changes, and
new schemes for exploiting systems are discovered, new rules need to be
developed, and existing rules may also need to be modified.

The objectives of an organization cannot be guaranteed by a simple
one-step procedure. Instead, we need to identify a network of
interconnected rules, some of which can be directly enforced to be
true. Examples of such rules include that users must authenticate, that
servers must possess a valid certificate, that a request must contain
a ticket with a valid signature, that the source and destination IP
addresses of a packet must occur in the list of valid combinations, and
so on. A combination of such conditions can be formulated which can be
enforced, and which also have the property that when they are all true,
the original objectives that we have been trying to ensure, must be true.

This process of finding conditions which can be enforced, to ensure
that risks are managed, and objectives attained, can always be
achieved. This cannot be proved. It can be checked, in individual
cases. In this dissertation,  a number of such experiments have been
undertaken, and in every case the necessary rules were found and
proved to be sufficient. Other examples have also been investigated in
 \cite{Hadaad15,sheniar2018experiments,sheniar2019Graph}.



\subsection{Mandatory Objectives}
In this dissertation,  we seek to illuminate and establish sound principles of security design. Hence, accepting the idea that there are design objectives which necessitate sacrificing services for users is not attractive, and at the very least seems to be a compromise that we should not accept.


In Operations Research, one of the most successful methodologies, over several decades, is optimization. In this framework, a problem is defined by two features: (i) an objective, and (ii) the constraints. These features define an optimization problem, the solution of which is to find the design which maximizes the objective without failing to meet the constraints.

This framework enables us to create a rigorous, scientific framework for many real world problems quite readily. We just have to find the constraints, and the objective, and then solve the optimization problem.


When we attempt to put security into this framework, it becomes clear that although it works, accurately, it fails to achieve the right emphasis. In security, the emphasis must be almost exclusively on the constraints. The constraints, in a security problem, are the rules: including business rules, service protection rules, and all the security rules required by the stakeholders. The objective will, in most cases, be obvious, but not particularly relevant.

It is usually felt that there are limited opportunities for achieving cost savings by searching for, or devising, better security strategies. There may be more such opportunities in the future, and this possibility should not, therefore, be entirely neglected. However, for the most part, the challenge, in security design, is not minimizing cost, or maximizing profit, but rather achieving certainty, that all required security rules have been enforced.

Many of the rules expected or mandated by stakeholders fall into fairly standard categories, like legal requirements, ethics, etc, which are fairly similar across different stakeholders. Most of these 
 requirements are well-known, but nevertheless, these requirements do call for rules that need to be enforced by security design. Therefore, it is useful for us to consider these requirements explicitly. We consider these generic rules first.

 
\subsection{Social Contract}
According to Hobbes \cite{Leviathan}, without a social contract, life is subject to continual fear, and danger of violent death; and the life of man is solitary, poor, nasty, brutish, and short. 
However, a social contract between the citizens of a society enables them to co-operate effectively, to achieve a better life without the need for constant fear.

Social contracts have been adopted in many modern nations, and in many cases, they appear to provide a good foundation for civil society, providing safety, wide-ranging freedom, with a high standard of living. Attempts to formulate an international social contract have been at best partially successful. There is an international court of justice, but many nations do not accept its jurisdiction \cite{Robertson}.

The Internet has made computing social  \cite{parameswaran2007social}. In turn, the Internet and the cyberspace based on it, is susceptible to a number of social ills that may befall any society, such as constant harassment, belligerent attacks on property, that is, cyberattacks, and invasions of privacy which seriously compromise the benefits for our social, educational, and commercial lives. By other words, with the convenience and intelligence offered by electronic agents inhabiting the web, and the world of apps, a crowd of unwanted denizens focused on criminality, exploitation, fraud, and chaos have taken the opportunities (of which there are many) to join us in cyberspace.  We must assume that whenever there is an opportunity for error, fraud, deception, or intimidation, it will be exploited. We need to actively defend against criminality, greed, parochialism, and chaos. Are these issues (which are surveyed in more detail in Section ??) due to the absence of a social contract? \\ In this study, We demonstrated the need for a social contract for cyberspace (and the Internet), i.e. a declaration of responsibilities and rights of all members of cyberspace, and also to propose a draft of such a social contract.
%\item \emph{proposed, demonstrated and developed by fair stakeholder policies that are formulated and agreed upon under the umbrella of a proposed social contract}, and then
%\item  \emph{designed and implemented by software that enforces those policies and their social contract and ensures  all parties are using this software}.
%\end{itemize}
%\end{flushleft}

\section{Significance of the Research}

Cyber security in this research, as it has not been previously, is:
%\begin{flushleft}
%\begin{itemize}


\section{Research Problem}
According to the philosophical and mathematical viewpoint of cybersecurity in this research:

Cyberspace can be considered as a network of heterogeneous entities
(individual users, organizations, devices, service providers, and
services etc.). These entities can be classified into two main categories,
the services, and their stakeholders (cyberspace diagram).  Stakeholders
can set their own policies regarding the services they engage with. To
assist them in this setting, they can use the policies declared by
those services.

Stakeholders polices are used to achieve their key (essential)
objectives. Different stakeholders have different objectives and also,
their objectives are subject to change.  For the stakeholders to be
able to work together effectively, their policies must be consistent.
Furthermore, as stakeholders objectives and policies change over time,
this consistency must be maintained.

Stakeholder objectives and policies are designed to maintain safety, privacy, security,
and any other key objectives. Any inconsistency between policies will lead
to a breakdown or compromise of the overall system.

In order for stakeholders to maintain their essential objectives, it is
generally necessary to develop a collection of enforcable conditions and to configure these conditions so that they achieve the objectives. This process is sometimes referred to as policy refinement. [Find citations in James Northway's
proposal.]

A social contract is needed to enable and manage these conditions (consistency diagram).  While ensuring the achievement of stakeholders’ objectives, the consistency of all the policies is essential, but this will be difficult unless stakeholders have a sound basis of shared {\em values}. The social contract provides this basis.
\section{Research Question(s)}
\if
What is the logical cybersecurity framework within which all the key concepts are consistent?\\
What is the cybersecurity framework within which all the key concepts are consistent?\\
What is the cybersecurity logical consistency framework within which all the key concepts are fit?\\
\fi
%What is the logical cybersecurity framework within which all the key concepts are consistent?\\

What is the logical process to achieve cybersecurity within which all the key concepts are consistent?
\begin{enumerate}
\item How can we develop and configure a collection of enforceable conditions to maintain safety, privacy, security, and any other key objectives?
\item How can we design, and implement software which enforces these conditions.  % and ensure that everyone installs thissoftware and uses only this software in the course of their interaction
\item Why is a social contract needed and how can one be proposed for cyberspace?
\end{enumerate}



\section{Research Aims and Objectives}
Cyber security policies should be produced within a teamwork approach.  This partnership needs a logical framework that controls these policies through a set of conditions that guarantee the achievement of the mandatory objectives.  The social contract comprehensively controls how these policies operate together, in other words, enables and manages these conditions.  This contract should be approved by the stakeholders. Finally, we develop, design, and implement software that enforces this contract, and ensure that this software is dealt with exclusively by all parties.
\begin{enumerate}
\item Demonstrating that cyberspace needs a social contract and proposing a draft of a social contract for cyberspace.
\item Developing a stakeholder security analysis which are: 
\begin{itemize}
\item Formulating the safety and security conditions, that are consistent with the social contract for cyberspace, as a statement of the user’s expectations of the system (Netml System).
\item Analysing the source code of the system to prove that it meets these conditions.
\end{itemize}
\item Developing Inference graphs, a graphical representation of cybersecurity architecture, which assist cybersecurity professionals to define the security objectives of their systems, the rules which enforce this security, and the reasoning behind the choice of these rules.
\end{enumerate}
\section{Research Limitations}
