\chapter{Literature Review}
\section{Cybersecurity Definition}
In recent years, the term cybersecurity has been widely used \cite{craigen2014defining,schatz2017towards,giles2013divided}......\\ 


However, because of its recent origin, there are differing interpretations of its meaning. Also, current interpretations of this term fluctuate quite dramatically for two reasons \cite{schatz2017towards}. The first is the diversity of sources of definition between governmental, professional and academic domains. The second  is the rapid development, especially in the field of technology, which sometimes leads to a change in the interpretation of key terms. 

In this section, we review  the existing literature to determine the definitions provided for the term cybersecurity by in current literature.
\if
In recent years, the term cybersecurity has been widely used\cite{craigen2014defining,schatz2017towards,gile 2013divided}, however, 
because of
	 its recent origin, there are differing interpretations of its meaning. 
	Also, current interpretations of this term fluctuate quite dramatically for two reasons \cite{schatz2017towards}. 
		The first is the diversity of sources of definition between governmental, professional and academic domains. 
		The second  is the rapid development, especially in the field of technology, which sometimes leads to a change in the 				 		interpretation of key terms.

 This diversity in assumed definitions leads to 
		a difference of views on what cybersecurity is and what are the related cybersecurity and ethical issues. 
		It is also likely to cause significant problems in the context of organizational strategies, stakeholders' policies consistency, business 			objectives, and international agreements.
\fi
 This diversity in assumed definitions leads to 
a difference of views on what cybersecurity is and what are the related cybersecurity and ethical issues. 
It is also likely to cause significant problems in the context of organizational strategies, stakeholders' policies consistency, business objectives, and international agreements.
\subsection{Governmental definitions}
 xxxxxxxxxxxxx
\subsection{Professional definitoins}
xxxxxxxxxxx
\subsection{Academic definitions}
xxxxxxxxxxxx
\\

This diversity in assumed definitions leads to a difference of views on what cybersecurity is and what are the related cybersecurity and ethical issues. It is also likely to cause significant problems in the context of organizational strategies, stakeholders policies consistency, business objectives, and international agreements. We proposed a new definition,In Chapter 1, with considerable advantages in clarity and simplicity, 
which is more appropriate for the research reported in this dissertation.

\subsubsection*{New Definition}
\begin{quote} \em Cybersecurity is the logical process, including design and implementation, of ensuring that one's mandatory objectives are met, where mandatory means the objectives which one wishes to be certain of achieving.\end{quote}

This definition of cybersecurity includes consideration of human values and ethics such as privacy, safety, and financial risks, ... anything that is mandatory. This view is adopted to allow the different mandatory goals
to be designed in a fully coordinated manner, which is logically necessary.

\section{Human Ethics and Values}
Fundamentally, human ethics and values should, or must, take precedence in our design of cybersecurity. The reason is that the whole system -- cyberspace -- is a human system. If it doesn't serve the real needs of humans, it must be changed so that it does[]. 

However, one of the most important challenges facing cybersecurity is the insertion of human values into software values and their maintenance.  We recognized  that as a resuls of three  main reasons below. 
\subsection {The rapid technological development}

\subsection  {Migration to Online activity}

As human activity migrates to an online form, human values need to move also.
 
\subsection {Cybercrime practices, unethical or illegal activities}

migration from physical to virtual domains

\section{ cybersecurity architecture}
\cite{Rerup2018}
%Related to Ch5...need to more details and update
\section{Knowledge Graphs for cybersecurity}
Cybersecurity systems have used Knowledge Graphs (KG) to store and  retrieve data  to make decisions about cyber-attacks 
\iffalse 
We believe that improving the base cyber threat intelligence representation will help improve the overall quality and performance of systems. The dependence of various cybersecurity informatics systems on various knowledge representation schemes makes it imperative that we develop systems that improve these representations. In some cases, a knowledge graph can have incorrect relationships between two cybersecurity entities, or may not even assert a relationship or a few missing relationships. In such a case, we can say that there are a few missing of relationships. we improve knowledge graphs by validating relationships and asserting values for missing relationships \fi \cite{pingle2019relext,jia2018practical, ghose2019multimodal, deng2019knowledge}. Graphical representation of logical or structural properties of computer systems is a well-established practice in computer science \cite{engelen2010integrating}. Inference graphs, as introduced and explained in chapters 3 and 5, provide a graphical representation of cybersecurity architecture.



